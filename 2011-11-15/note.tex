\documentclass[12pt]{article}
\usepackage{parskip}
\usepackage{enumerate}
\usepackage{listings}
\usepackage{fullpage}

\begin{document}

\title{PSYCH 101}
\author{Matthew Visser}
\date{Nov 15, 2011}
\maketitle

\section{Personality}

\subsection{Introduction}

We are going to look at shyness. We look at \emph{Trait vs. State}. Some people
may be shy only in certain conditions (state), while others are always shy
(trait). For something to have an affect on \emph{personality}, it must be a
trait.

Most of this is taken from Zombardo.

\begin{tabular}{|p{0.5\textwidth}|p{0.5\textwidth}|}
	\hline
	STATE SHYNESS & TRAIT SHYNESS\\
	\hline
	80\% of population & 25\% of population \\
	\hline
	Shy in certain situations & Pervasive shyness: \begin{enumerate}
		\item across situations
		\item across time
	\end{enumerate} \\
	\hline
	Minimal negative consequences and possibly positive consequences
	(\textit{i.e.} may elicit empathetic response). & Severe consequence
	\begin{itemize}
		\item enjoying social events and meeting new people
		\item makes other people uncomfortable
		\item inhibits communication, expression, and assertiveness
		\item promotes excessive self-consciousness and self-preoccupation
			(\textit{i.e.}, introversion)
		\item prone to low self-esteem, stress, loneliness, and depression
		\item self-perpetuating
	\end{itemize}\\
	\hline
\end{tabular}

\subsection{Psychodynamic View}

\begin{itemize}
	\item To the psychodynamic view, personality is always a conflict in the
		subconcious mind.
	\item Either repression or lack of stimulus in some phase of
		development
\end{itemize}

\subsection{Behavioural View}

\begin{itemize}
	\item You learn certain behaviours, and some stick. This turns into a trait.
	\item When you have certain behaviours and are rewarded for them, you
		develop them as traits.
	\item Believe that you can develop behaviour that counters current
		behaviours, \textit{e.g.} if someone is shy you can condition them to be
		more outgoing.
\end{itemize}

\subsection{Social-Cognitive View}

\begin{itemize}
	\item Sources of efficacy is accomplishment.
\end{itemize}

\subsection{Humanistic View}

\subsection{Trait View}

We view some personality traits and take statistics on them to view how people
with certain traits perform in certain environments with certain tasks.

\begin{itemize}
	\item Extraversion
	\item Emotional Stability
	\item Openness to experience
	\item Agreeableness
	\item Concientiousness
\end{itemize}

\section{Social Psychology}

\subsection{Persuasion}

\begin{itemize}
	\item We can use relationships to convince people.
	\item We want to change someone elses beliefs, behaviour or attitudes.
	\item We want to give someone positive facts about something to form a
		positive opinion.
	\item Can use behaviour modification with conditioning.
\end{itemize}<++>

\subsection{Conformity}

\begin{itemize}
	\item Informational Conformity
		\begin{itemize}
			\item More likely to give same answers as other people when you are subject
				to other's answers.
		\end{itemize}
	\item Normative Conformity
\end{itemize}

\subsection{Compliance}

Usually, voluntary behaviour change.

\subsection{Obedience to Authority}

Involuntary behaviour change.


\end{document}
% vim: tw=80
