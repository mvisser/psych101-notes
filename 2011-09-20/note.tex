\documentclass[12pt]{article}
\usepackage{parskip}
\usepackage{enumerate}
\usepackage{listings}
\usepackage{hyperref}
\usepackage{fullpage}

\begin{document}

\title{PSYCH 101}
\author{Matthew Visser}
\date{Sep 20, 2011}
\maketitle

\textbf{Song of the day:}
\url{http://lyrics.wikia.com/Five_For_Fighting:100_Years}

\section{Developmental Psychology}

\begin{itemize}
    \item Physical Development
        \begin{itemize}
            \item Early and late in life -- big changes
        \end{itemize}
    \item Cognitive Development
    \item Social Development
\end{itemize}

\subsection{Physical Development}

\subsubsection{Infancy}

\begin{itemize}
    \item Progression along different stages of muscle control. Generally happen
        along a certain order (see diagram in textbook).
    \item not the same for every child, order not necessarily fixed.
    \item We see the opposite happen when people get older. People get vision
        problems, muscles deteriorate, \textit{etc.}
\end{itemize}

\subsection{Cognitive Development}

\subsubsection{Piaget's Theory}

\begin{itemize}
    \item Birth to nearly 2 years --- \textit{Sensorimotor}
        \begin{itemize}
            \item Object permanence
            \item Stranger anxiety
            \item experience world through senses
        \end{itemize}
    \item 2--6 years --- \textit{Pre-operational}
        \begin{itemize}
            \item Representing things with words and images but lack reasoning
            \item Pretend play
            \item egocentric -- think everyone sees the world like they do
            \item language development -- can learn any language and grammar
                rules very easily at this age
            \item Assimilate things into schemas
        \end{itemize}
    \item 7--11 years --- \textit{Concrete operational}
        \begin{itemize}
            \item Conservation -- realize the same quantities in different
                containers
            \item mathematical transformations
            \item Thinking logically about concrete operations
            \item grasping analogies and arithmetic
            \item Ideas about ideas
        \end{itemize}
    \item 12--Adulthood --- \textit{Formal Operation}
        \begin{itemize}
            \item Abstract thought
            \item morals and moral reasoning
        \end{itemize}
\end{itemize}


\subsubsection{ Extensions and Qualifications of Piaget's Theory}

\begin{itemize}
    \item Does object permanence really emerge at or around 8 months?
    \item Egocentrics and research on theory of mind.
    \item Development is more continuous and happens earlier than Piaget
        proposed.
\end{itemize}

Research at UW by Dr.\ Daniela O'Neill:
\begin{itemize}
    \item Study 1
        \begin{itemize}
            \item The participants were 2 and a half years old
            \item The experiment hid a toy in one of several locations when the
                parent was looking or not
            \item The child had to direct the parent to the location
            \item The child used more gestures and named the location more when
                the parent wasn't looking when the item was hid
            \item Parent got more help when they weren't looking.
        \end{itemize}
    \item Study 2
        \begin{itemize}
            \item Participants were 2 years and 3 months old.
            \item Similar to study 1 but with stickers in containers.
            \item Children used more gestures when parents weren't looking.
        \end{itemize}
\end{itemize}

\subsection{Social Development}

\subsubsection{Attachment Theory}

\begin{itemize}
    \item Harlow's Monkey studies
    \item The theory behind attachment theory
    \item The methodology behind attachment theory
    \item What attachment predicts
    \item social competence
    \item even good adult relationships
\end{itemize}

\subsubsection{Erikson's Theory}

See table 4.2 in text.

\subsubsection{Gender Development}

\begin{itemize}
    \item Common observation is that children's behaviour and particularly
        play is gender stereotyped. Why?
    \item Is it the way their parents treat them? Parent's attitudes actually
        has no influence on how children view gender roles.
    \item Two theories of gender typing.
        \begin{itemize}
            \item See how other kids and people in society at large act and
                follow it.
            \item Cultural learning of gender $\to$ gender schema looking at
                self and world through gender ``lens'' $\to$ gender-organized
                thinking \& gender typed behavior
        \end{itemize}
\end{itemize}


\end{document}
% vim: tw=80
