\documentclass[12pt]{article}
\usepackage{parskip}
\usepackage{enumerate}
\usepackage{listings}
\usepackage{fullpage}

\begin{document}

\title{PSYCH 101}
\author{Matthew Visser}
\date{Sep 13, 2011}
\maketitle

\section*{Introduction}

\subsection*{Reasearch Experiences Group}

You can take part in experiments in exchange for course grades. Can read about
research methodology in text, but can also participate in studies.

\begin{description}
    \item[Research participation:] 4\%, up to 6\%
    \item[Course work:] 96\%
\end{description}

Can sign up by logging in to REG web site. Can also get details from outline.
UID=``Quest user Id'', password is Quest password.

Of the four credits you can get, only half can be from online studies.

Pre-screening test screens for tests. Mass testing is an actual test for credit.
The mass testing survey ends in October 5.

\textbf{Credits} are earned through time, so 30 mins are a half credit.
You can also get credits by doing an article review. \textbf{Aside}: do an
article review on placebo tests.

\section*{History of Psychology}
\subsection*{Three Traditions}

Who was the first psychologist?\\
\textit{Sigmund Freud, Wilhelm Wundt, and Sir Francis Galton}

There isn't one beginning to psychology. You could trace psychology to back to
philosophers such as Plato.

Modern psychology does not necessarily come from Freud, although he was popular.
Freud was like the Dr.\ Phil of his age.

The three traditions are:
\begin{itemize}
    \item Psychiatric
        \begin{itemize}
            \item Freud and psychoanalysis
            \item Started as case studies, \textit{e.g.}\ A lady with glove numbness.
        \end{itemize}
    \item Testing
        \begin{itemize}
            \item Galton, Binet and intelligence testing
        \end{itemize}
    \item Academic
        \begin{itemize}
            \item Wundt, James, Watson, Gestalt - lab psychology.
        \end{itemize}
\end{itemize}

\subsection*{Psychiatric Tradition}

The major theorists of the psychiatric tradition are:
\begin{itemize}
    \item Freud
        \begin{itemize}
            \item Motivations are: sex, aggression.
        \end{itemize}
    \item Jung
        \begin{itemize}
            \item Thought people were much more complex.
            \item Very influential in literature.
        \end{itemize}
    \item Adler
        \begin{itemize}
            \item Said it was power that drives people
        \end{itemize}
    \item Ego Psychologists
        \begin{itemize}
            \item Came after Freud
            \item made Freud's theories more palatable to psychologists.
            \item Includes Eric Ericson, who was Freud's chauffeur.
            \item Includes Horney
            \item Includes Sullivan
                \begin{itemize}
                    \item speculated on attachments between parents and infants.
                \end{itemize}
        \end{itemize}
\end{itemize}

The major assumptions of the psychiatric tradition are:
\begin{itemize}
    \item Distinguish between concious and subconscious or unconscious.
    \item Can't rely on what people say or do to indicate problems.
        \begin{itemize}
            \item much or most of mental life is unconscious.
        \end{itemize}
    \item Must understand people comprehensively as a dynamic whole.
        \begin{itemize}
            \item can't understand just a part of an individual.
        \end{itemize}
    \item Best approach is clinical
        \begin{itemize}
            \item More is learned by studying those who are sick.
        \end{itemize}
\end{itemize}

The testing traditions of the psychiatric tradition are:
\begin{itemize}
    \item Galton
        \begin{itemize}
            \item Assumed he could sort people by high and low class through
                testing intelligence.
            \item cousin of Darwin, thought he could select genetically superior
                humans.
        \end{itemize}
    \item Spearman
        \begin{itemize}
            \item A student of Galton
            \item developed statistical testing for psychology.
        \end{itemize}
    \item Cattel and Eysenck
        \begin{itemize}
            \item Tried to figure out the important dimensions of personality.
            \item recognized introversion vs extroversion
            \item Anxiety or being laid back.
        \end{itemize}
    \item Binet
        \begin{itemize}
            \item Developed the first working intelligence test in France.
            \item had a simpler approach to intelligence testing.
        \end{itemize}
    \item Most of the testing traditions are trying to prove a genetic
        superiority.
\end{itemize}

\subsection*{Testing Tradition}

Major assumptions of this tradition are:
\begin{itemize}
    \item Psychology is primarily a product of biology
        \begin{itemize}
            \item Emphasis on nature over nurture \& evolution
        \end{itemize}
    \item There are only a few ways that people differ
        \begin{itemize}
            \item A small number of traits can explain differences
        \end{itemize}
    \item Peoples traits can be understood through simple tests.
        \begin{itemize}
            \item paper and pencil tests are the way people are studied
        \end{itemize}
\end{itemize}


\subsection*{Academic Tradition}

The major theorits of this tradition are:
\begin{itemize}
    \item Wundt -- Atomism
        \begin{itemize}
            \item Wanted to come up basic units of thought, like a periodic
                table
            \item Started a scientific methodology for testing psychology.
        \end{itemize}
    \item William James -- Pragmatism
        \begin{itemize}
            \item Brought psychology from Europe to North America at the turn of
                the 20\textsuperscript{th} century.
            \item Wrote the first textbook on psychology.
            \item Used to be a professor of philosophy.
            \item Wanted to study what people care about.
                \begin{itemize}
                    \item Emotions
                    \item basic thought
                    \item memory
                    \item what people think
                \end{itemize}
            \item Brought psychology to undergraduate.
        \end{itemize}
    \item Watson, Skinner -- Behaviorism
        \begin{itemize}
            \item Observe behaviours, not thoughts
            \item can't study thoughts, but can study behaviors
            \item their thoughts dominated psychology from 1920's - 1950's
        \end{itemize}
    \item Lewin -- Gestalt
        \begin{itemize}
            \item Thinking about thoughts and perceptions
            \item interested in how people perceive each other.
            \item moved from Europe during Hitler's rule to North America
        \end{itemize}
\end{itemize}

The major assumptions of the academic tradition are:
\begin{itemize}
    \item Psychology is best understood when causes are understood,
        \textit{i.e.}\ what
        causes things to happen.
        \begin{itemize}
            \item The \textit{why} is more important than the \textit{how} or
                \textit{when}.
        \end{itemize}
    \item People's thoughts and actions can be understood
        \begin{itemize}
            \item By observing people we can understand them
        \end{itemize}
    \item Systematic study produces the fullest understanding.
        \begin{itemize}
            \item Set up labs and experiments.
        \end{itemize}
\end{itemize}

\subsection*{Three new developments in Psychology}

\begin{itemize}
    \item Psychiatric tradition
        \begin{itemize}
            \item Humanistic approach
                \begin{itemize}
                    \item Emphasizes the goodness of people, treating people
                        well, and what people are capable of.
                \end{itemize}
        \end{itemize}
    \item Testing Tradition
        \begin{itemize}
            \item Development of the Big 5 model of Personality (OCEAN)
                \begin{description}
                    \item[O] -- Openness
                    \item[C] -- Conscientiousness: concious of doing things
                        early
                    \item[E] -- Extroversion: whether you are outgoing or shy
                    \item[A] -- Agreeable-ness: Whether you go along with
                        things, or are firm and stand their ground no matter
                        what.
                    \item[N] -- Neuroticism: the sense of feeling anxiety.
                \end{description}
        \end{itemize}
    \item Academic Tradition
        \begin{itemize}
            \item Cognitive Revolution
                \begin{itemize}
                    \item In North America behaviourism dominated, but changed
                        in the 1950's.
                    \item The importance of thought emerged. People started to
                        study thoughts.
                \end{itemize}
        \end{itemize}
\end{itemize}

\section*{Research Methodology}

There is a basic methodology in how the world should be understood in science.
Most people agree with this approach, but not everybody. There are some
assumptions.
Science didn't develop as a systematic approach to understand the world until
the 18\textsuperscript{th} century.

\textbf{Our ideas can be wrong.}

Take for example Newton's laws, which were overshadowed by Einstein's theories.
The scientific approach recognizes that their ideas could be wrong, and then
demonstrate your ideas are right or wrong.

\textbf{Our ideas may seem right now, but might actually be wrong later.}

We can accept ideas now if they are proven, but it may later be demonstrated to
be incorrect later.

\textbf{What is accepted is influenced by what is popular.}

What is popular now may be thought about a lot, but other things could be
missed.

\textbf{What is accepted is shaped by political forces and has political
impact.}

What is accepted as true impacts how we think and \textit{vice versa}.

\subsection*{Ways of Doing Research}

\begin{itemize}
    \item Tension between discovery and explaination.
        \begin{itemize}
            \item Galileo and planetary motions around sun
            \item Einstein and $E = mc^2$
        \end{itemize}
    \item Case Study
        \begin{itemize}
            \item Anna O.
        \end{itemize}
    \item Naturalistic Observation
        \begin{itemize}
            \item Jane Goodall's research with Chimps
                \begin{itemize}
                    \item use tools
                    \item have language
                    \item mating rituals
                \end{itemize}
        \end{itemize}
    \item Survey
        \begin{itemize}
            \item Establishes generality and \textit{Correlations}.
            \item Using statistics and proper random selection, you can
                understand a lot about a population from a relatively small
                subset.
        \end{itemize}
    \item Experiment
        \begin{itemize}
            \item Establishes \textit{Causation}.
            \item Often more explanation, but sometimes large discovery.
        \end{itemize}
\end{itemize}

\subsection*{Surveys and Establishing Representativeness}
\begin{itemize}
    \item Representative group of people
        \begin{itemize}
            \item Random sampling: take a small subset of people that are evenly
                distributed across populations.
        \end{itemize}
    \item Ask unbiased questions
        \begin{itemize}
            \item Some surveys lead people to answer one way or another.
        \end{itemize}
    \item Establishes Correlation among variables.
\end{itemize}

\subsection*{Correlation is not Causation - How To Establish Causation}

\begin{itemize}
    \item Correlation
        \begin{itemize}
            \item When one variable changes, the other variable changes.
        \end{itemize}
    \item Time order
        \begin{itemize}
            \item If we know two things are related, but not which is first, we
                don't know which causes which.
        \end{itemize}
    \item Elimination of all other possible causes
        \begin{itemize}
            \item If there is another possible cause, two related events may not
                cause one another.
            \item get away from this with random assignment
        \end{itemize}
\end{itemize}

\subsection*{How Experiments Try to Show Causation}

\begin{itemize}
    \item Independent Variables
        \begin{itemize}
            \item Specific way of describing cause.
            \item Need to isolate difference to only the independent variables.
        \end{itemize}
    \item Dependent Variables
        \begin{itemize}
            \item Need to have a way to test if the dependant variable exists
                (and do so ethically).
            \item Need this to be valid, \textit{i.e.}\ are you measuring what
                you say you are?
            \item Reliable -- no false positives or negatives.
        \end{itemize}
    \item Time Order is built in.
        \begin{itemize}
            \item Alter independent variable before dependent.
        \end{itemize}
    \item Eliminating other causes
        \begin{itemize}
            \item Random assignment
        \end{itemize}
\end{itemize}


\end{document}
% vim: tw=80
