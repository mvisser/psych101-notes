\documentclass[12pt]{article}
\usepackage{parskip}
\usepackage{enumerate}
\usepackage{listings}
\usepackage{fullpage}

\begin{document}

\title{PSYCH 101}
\author{Matthew Visser}
\date{Sep 27, 2011}
\maketitle

\textbf{Song of the day:} ``My Poor Brain'' by Foo Fighters.

\section{The Nervous System and the Brain}

\subsection{The Big Picture}

\begin{itemize}
    \item Neuron --- the basic unit of the brain. They have different shapes and
        functions, but they are the building blocks
    \item The peripheral nervous system --- nerves outside the brain and spine to
        control movement and gather input/senses.
    \item The Central Nervous System and the Brain --- how the multiple systems
        co-ordinate and work together.
    \item The Endocrine System --- part of how the nervous system works. This is
        formed by hormones that affect the nervous system.
    \item Putting everything together using hunger as an example.
\end{itemize}

\subsection{The Neuron}

See a diagram of the neuron to see the pieces.

\subsection{Neurotransmitters}

\begin{itemize}
    \item Dopamine --- involved in movement, attention, emotion. Deficiencies of
        it in Parkinson's Disease. Excesses of it in Schizophrenia. Dopamine
        controls smooth movement of muscles.
    \item Serotonin --- involved in mood, hunger, sleep, arousal. Deficiencies
        are associated with depression.
    \item Norepinephrine --- involved in alertness and arousal. Deficiencies
        associated with depression.
    \item Acetylcholine --- involved in muscle movement, learning and memory.
        Damage to neurons that produce it is part of Alzheimer's. If you don't
        have this, your muscles don't work.
    \item Endorphins --- involved in pain regulation. Heroine and morphine mimic
        this neurotransmitter. Drugs called this because they mimic the effect of
        endorphins.
\end{itemize}

\subsection{Divisions of the Nervous System}

The peripheral nervous system is divided into:
\begin{itemize}
    \item Automatic -- divided into sympathetic  and parasympathetic
        (preperation)
    \item Somatic
\end{itemize}

\subsection{The Central Nervous System}

\begin{itemize}
    \item Spinal cord and brain.
    \item Spine carries information and controls some reflexes.
    \item The brain does almost all processing of information. There are 3 main
        areas of the brain to know about:
        \begin{itemize}
            \item Brain stem --- controls basic functions like making the heart
                beat or making you breathe.
            \item  Lymbic System --- Controls motivation and memory. Located
                around brain stem.
            \item The Cerebral cortex --- complex processing. This is not
                necessarily needed to live. If there is damage to this section,
                other parts of it could pick up after the damage. There is some
                redundancy in this section.
        \end{itemize}
    \item The Cerebellum controls a lot of fine muscle control and movement.
\end{itemize}

\subsection{Cerebral Cortex Functional Divisions}

\begin{itemize}
    \item The Corpus Callosum seperates the two sides of the brain. When
        severed, they function differently for a while.
    \item People who have the corpus callosum severed are called split brain
        patients. These people can control different limbs on different sides
        independently.
    \item Left side is a lot more verbal.
    \item Right side is much more verbal.
\end{itemize}

\subsection{The Endocrine System}

\begin{itemize}
    \item Hormones are controlled by glands such as the Pituitary gland,
        thyroid, and pancreas.
\end{itemize}

\subsection{Case Study --- Hunger}

What makes us feel hungry?
\begin{itemize}
    \item Stomach contractions
    \item blood sugar levels
    \item stimulated lateral hypothalamus releases the hormone orexin which
        heightens hunger.
\end{itemize}

When we eat, what makes us not feel hungry?
\begin{itemize}
    \item Stomach expands and this is transmitted to the hypothalamus.
    \item Blood sugar levels interpreted, insulin released.
    \item Ventromedial hypothalamus is stimulated and leads to the feeling of
        satiation.
\end{itemize}

\subsection{Damage to the Hypothalamus and Rats}

\begin{itemize}
    \item If there is a lesion to the lateral hypothalamus then rats starve
        themselves to death. They also seem sleepy and depressed.
    \item If the ventromedial hypothalamus is damaged then the rats will never
        stop eating.
\end{itemize}

\subsection{Overarching Principles}

\begin{itemize}
    \item There is a lot of specificity or specialization in function
    \item structures of the brain and the various systems work together in
        coordinated, overlapping and redundant systems
    \item Despite all this specificity and complexity, the brain has quite a bit
        of plasticity. This means if an area is damaged other areas can take
        over --- particularly among young children. This allows people with
        brain damage to still function normally, depending on where the damage
        is and how much.
\end{itemize}

\hrule

\textbf{Second song of the day}: ``Go To the Mirror Boy'' by The Who.

\section{Co-ordinated Brain Activity}

\begin{itemize}
    \item The senses
        \begin{itemize}
            \item Vision
            \item Hearing
            \item Others --- taste, smell, touch, pain, balance
        \end{itemize}
    \item Language processing in the brain
    \item Your brain while you're sleeping
\end{itemize}

\subsection{Vision}

\begin{itemize}
    \item The parts of the eye can be seen on the slides.  Fovea is the place
        where the eye can see the best.
    \item The eye has many cones and rods.  Rods are sensitive to low levels of
        light.  Cones are receptive to colours, while rods are not. There are
        three types of cones: red, green and blue.
    \item The nerves for each eye are connected to the thalamus of the opposite
        side of the brain. This then passes the information to the visual cortex
        in the occipital lobe.
    \item In the cortex, individual cells respond to lines at various angles.
    \item In the cortex, individual cells respond to lines at various angles.
    \item Cells converge to create perception of what we see.
    \item Higher level cells provide more interpretation
    \item All of this happens with a lot of parallel processing.
    \item Trichromatic theory
        \begin{itemize}
            \item We can see colour because of the three cones.
            \item The cones mixed to make all of the colours.
            \item The cons are consistent for colour blind people.
        \end{itemize}
    \item Opponent-process theory
        \begin{itemize}
            \item Have three opponent colour pairs --- black and white, red and
                green, blue and yellow
            \item This explains afterimage effects.
        \end{itemize}
    \item Both the trichromatic and opponent-process theories are probably true.
\end{itemize}

\subsection{Hearing}

\begin{itemize}
    \item Sounds waves stimulate hearing.
    \item Waves caught by outer ear
    \item First transformed into a physical impulse by the ear drum.
    \item The drum moves bones which move fluid in the Cochlea, which then
        transforms the movement to nervous pulses.
    \item The hammer, anvil and stirrup move the oval window, which moves fluid
        in the cochlea. This fluid moves small hairs in the cochlea in the
        basilar membrane.
    \item The Place theory says that it is the place on the basilar membrane
        that is stimulated that creates the pitch
    \item The frequency theory says that it is the speed of the movement of the
        fluid in the cochlea that determines the pitch.
    \item Both theories seem to be true, place theory explains high pitches,
        frequency explains low pitches.
\end{itemize}

\subsection{Taste}

Four basic sensations are
\begin{itemize}
    \item sweet
    \item sour
    \item salty
    \item bitter
\end{itemize}

The sensations are caused by a chemical reaction on your taste buds. Taste is
heavily influenced by smell, which is called sensory interaction.

\subsection{Smell}

\begin{itemize}
    \item A chemical sense
    \item Reactions in the olfactory membrane that create smells
    \item Still don't understand how receptors work
    \item odors can evoke memories and emotions
\end{itemize}

\subsection{Touch}

\begin{itemize}
    \item Specialized nerve endings in skin, but they don't have a 
        relationship with what we feel.
    \item Skin is sensitive to pressure, heat, and pain.
    \item  The sensitivities give rise to quite varied sensations.
    \item hot is warm + cold
    \item Feeling something wet is cold and pressure at the same time.
\end{itemize}

\subsection{Pain}

\begin{itemize}
    \item We know much more about pain, but the more that we know, the more
        difficult it is to understand
    \item There is no single area in the brain that controls pain.
    \item Gate-Control Theory --- the spinal chord has a gate that either
        blocks pain signals or allows them to travel to the brain. Distractions
        or endorphins can close the gates.
\end{itemize}

\subsection{Balance}

Balance is called the \emph{Kinesthetic sense}. Can experience this by
\begin{itemize}
    \item Joints and muscles telling us position of body
    \item Semi-circular canals in ears tell us vestibular sense or position of
        our head.
\end{itemize}

\section{Synaesthesia}

This is:
\begin{itemize}
    \item The blending of senses.
    \item An example is digit-colour synaesthesia.
\end{itemize}

\end{document}
% vim: tw=80
