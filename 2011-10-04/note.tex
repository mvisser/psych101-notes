\documentclass[12pt]{article}
\usepackage{parskip}
\usepackage{enumerate}
\usepackage{listings}
\usepackage{fullpage}

\begin{document}

\title{PSYCH 101}
\author{Matthew Visser}
\date{Oct  4, 2011}
\maketitle

\textbf{Song of the day}: ``Dreaming in Metaphors'' by Seal.

\section{Co-ordination of the Brain}

\subsection{Speech}
\begin{itemize}
	\item Wernicke's area interprets auditory input.
	\item Broca's area controls speech, converts thoughts to speech, starts mouth
		movements.
\end{itemize}

Other areas of the brain can take over for the 5 main areas for language if the
original ones are damaged.


\section{Sleep and Dreaming}

The brain is actually very active while we are sleeping. There are five stages
of sleep defined by brain activity. REM sleep is when we dream.
\begin{itemize}
	\item REM sleep has very similar brain activity to when we are awake.
	\item When someone is woken up when their eyes are moving a lot, they report
		they are dreaming.
	\item Almost all of the time, people are immobilized during REM sleep, and
		the brain is completely cut off from physical movement (except the eyes
		of course).
	\item Sleepwalking usually happens in stage 4 (need to verify).
	\item Sleep paralysis happens when you wake up but your brain hasn't
		regained attachment to the rest of the body.
	\item You have more REM sleep later in the night, and less stage 4 sleep
		later in the night.
\end{itemize}

Why sleep?

\begin{itemize}
	\item We feel terrible when we don't.
	\item When we don't sleep, we aren't alert and are more likely to get into
		accidents.
\end{itemize}

Why dream?

\begin{itemize}
	\item Freud says wish fulfillment.
	\item Modern theory is called ``activation-synthesis'' -- the experiences we
		had during the day get played out in random order. This helps to lay
		down memories. All of the activity gets turned into creative and
		meaningful memories and thoughts.
	\item The evidence for activation-synthesis is that babies have a lot more
		REM sleep and dreams than adults. Also, people who have less REM sleep
		do more poorly on tests. The brain also adds more REM sleep if you
		haven't had enough, this is called REM rebound.
	\item Probably need dreams for memory consolidation and cognitive
		development.
\end{itemize}

\vspace{1cm}
\hrule

\textbf{Second song of the day}: ``Illusions in G Major'' by Electric Light
Orchestra.

\section{Perception, Preconceptions and Judgement}

\begin{itemize}
	\item Perception --- the whole is not equal to the sum of its parts.
		\begin{itemize}
			\item Figure and Ground in perception
			\item Gestalt Principles
		\end{itemize}
\end{itemize}

Gestalt psychologists argued:
\begin{itemize}
	\item whole is not sum of its parts
	\item By this they mean our thoughts and preconceptions shape what we
		perceive
	\item The point is we do not see exactly what is there, instead our
		perceptions are shaped by thoughts.
\end{itemize}

Figure and Ground in perception is when you can see a picture in two different
ways. See slide 10.

We tend to group things based on 
\begin{itemize}
	\item proximity
	\item similarity
	\item continuity
	\item connectedness
\end{itemize}

What we are exposed to can change what we perceive.

\end{document}
% vim: tw=80
