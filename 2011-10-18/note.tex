\documentclass[12pt]{article}
\usepackage{parskip}
\usepackage{enumerate}
\usepackage{listings}
\usepackage{fullpage}

\begin{document}

\title{PSYCH 101}
\author{Matthew Visser}
\date{Oct 18, 2011}
\maketitle

\textbf{Song of the Day}: \emph{Brian Wilson} by the Barenaked Ladies.

\section{Learning}

\subsection{Behaviourism}

\begin{itemize}
	\item Classical conditioning -- Pavlov and dogs
	\item Operant Conditioning -- Skinner and pigeons
	\item Language Acquisition
	\item Observational Learning -- The Bobo Doll
\end{itemize}

\subsection{Pavlov}

Pavlov studied spit and saliva. Altered the dogs mouths to collect the spit.
Pavlov noticed that dogs salivated only once the food came out and was set in
front of them. Then he noticed that when he started to walk across the room, the
sound of his steps made the dogs salivate, even when he wasn't coming to feed
them. Pavlov then tried a bunch of different triggers --- a bell, tuning fork or
bright light.

There are some terms to this phenomena.
\begin{itemize}
	\item Unconditioned stimulus --- a stimulus that gets a response naturally.
	\item unconditioned response --- a natural response to a stimulus.
	\item Conditioned response --- a response that occurs because of
		conditioning.
\end{itemize}

\begin{itemize}
	\item In \emph{classical conditioning}, learning is the association of a CS
		with a UCS to the extent that the CS produces the UCR (now it is a CR).  
	\item Generalization occurs when the CS is changed slightly and the CR still
		happens. 
	\item Discrimination occurs when the CS is changed and the CR does not
		happen.
	\item Extinction occurs when the CS no longer produces the UCR.
\end{itemize}

\subsection{The Beginning of Behaviourism}

In 1920, Watson gave the claim that he could take 20 children and make them grow
up to anything.

Watson conditioned a child called Little Albert, clapping his hands loudly
behind him every time a rat or bunny came close to him. Albert was then
conditioned to be afraid of these animals.

\subsection{Operant Conditioning}

Promises to explain all behaviors that organisms do that they wouldn't naturally
do.

\begin{itemize}
	\item How does an organism learn to do things it normally wouldn't?
	\item Operant conditioning has the answer: behaviour that is rewarded will
		occur more often.
	\item Complex behaviours can be shaped through rewards.
	\item Learning is the acquisition of new behaviours.
	\item Positive reinforcer --- getting good things.
	\item Negative reinforcer --- removing bad things, \emph{not} a punishment.
	\item Primary reinforcers --- pleasant in their own right.
	\item Conditioned reinforcers --- pleasant through their association (note
		the classical conditioning here).
\end{itemize}

\subsection{Language Acquisition}

\begin{itemize}
	\item Skinner asserted that we learn language through reinforcement.
	\item Chomsky thought we have an inborn ability to learn language.
	\item Behaviourism doesn't make sense, because it's impossible to condition
		so many words a day.
	\item Childrem pick up grammar rules so well that they overapply rules in
		ways they weren't supposed to.
\end{itemize}

\subsection{Observational Learning}

\begin{itemize}
	\item Can we / do we learn by watching others?
	\item Do we only learn by being rewarded?
	\item Bandura thought that perhaps we can learn just by watching. Note how
		this requires thought to intervene.
	\item Consider the Bobo Doll stody.
		\begin{itemize}
			\item An adult beats up a Bobo Doll. While the child watches.
			\item The child beats up the Bobo Doll when they have observed an
				adult doing so. They also extended the behaviour by devising new
				ways of hitting the doll.
		\end{itemize}
\end{itemize}

\vspace{1cm}
\hrule

\section{The big Picture --- Fundamental Issues}

\begin{itemize}
	\item Nature vs. Nurture
	\item How applicable is psychology to daily life.
	\item Is there still a role for behaviourism?
\end{itemize}

\subsection{Schemas and Heuristics}

Hearing someone speak about something without context gets confusing. You need
some sort of schema or context. A \emph{schema} is a way we think about
something, but isn't a definition.

\begin{itemize}
	\item Representativeness Heuristic ---
		When asked  about connections between some information, we think about
		what fits better rather than logically thinking.
	\item Availability Heuristic --- We only really take into account
		information that remember better.
	\item Anchoring and Adjustment --- An initial proposition of a number gets
		people to guess numbers closer to the proposed number, rather than what
		is more logical.
	\item risky choices
		\begin{itemize}
			\item We have risk aversion. If given a 85\% chance gamble of \$100
				or \$85 for sure, we pick the money for sure
			\item We increase risk when it is equal, but favour less loss.
		\end{itemize}
\end{itemize}

\subsection{Some UW Research By Dr.\ Jeniffer Stolz}

\begin{itemize}
	\item For half participants, each trial begins with a very brief prime (for
		example, bacon).
	\item For the other half, there are no primes.
	\item Next, two different words come on the screen at almost the same time.
		One word is related to the prime (for example, eggs), the other word is
		unrelated to the prime (for example, water).
	\item People perceive the primed word first if primed, otherwise they are
		equally likely to judge the first one.
\end{itemize}


\end{document}
% vim: tw=80
