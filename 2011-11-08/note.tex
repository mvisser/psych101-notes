\documentclass[12pt]{article}
\usepackage{parskip}
\usepackage{enumerate}
\usepackage{listings}
\usepackage{fullpage}
\usepackage{array}

\begin{document}

\title{PSYCH 101}
\author{Matthew Visser}
\date{Nov  8, 2011}
\maketitle

\section{Freud}

\begin{description}
	\item[Id:] The animal instincts are embodied in the Id.  
		\begin{itemize}
			\item The Id is driven by two opposing instincts:
				\begin{description}
					\item[Eros:] the life force. influances our sexual development.
					\item[Thanatos:] The death wish, played out with aggression.
				\end{description}
			\item Has desires for for sex and aggression.
		\end{itemize}
	\item[Ego:] Our realistic principle. Relizes that we need compromise.
		\begin{itemize}
			\item Caught between the Id and the Superego.
		\end{itemize}
	\item[Superego:] the ideal ego, tries to make you the perfect person.
\end{description}

\subsection{Child Development}

The \emph{sapling theory} assumed that children were just small adults, just
lacking experience. Freud asserted that children did not develop like that, and
had special needs.  Freud's stages of psychosexual development are:
\begin{enumerate}
	\item Oral --- 0-18 months. Pleasure centers on the mouth.
	\item Anal --- 18-36 months. Pleasure focusses on bowel/bladder elimination;
		coping with demands for control.
	\item Phallic --- 3-6 years. Pleasure zone in the geniatls
		\begin{description}
			\item[Oedipus Stage] Notices genitals.
				\begin{itemize}
					\item Boys notice differences in genitals.
					\item Develop incestuous sexual attraction for their
						opposite gendered parent.
					\item Realize there is competition, the same-gendered
						parent.
					\item Realizes a fear of castration and tries to become like
						the same-gendered parent to gain their approval and
						become attractive.
				\end{itemize}
			\item[The Electra Complex] Similar to Oedepis.
				\begin{itemize}
					\item Take on feminine characteristics because of an
						incestuous attraction to the opposite-gendered parent.
					\item Realizes that she lacks a penis, develops penis envy.
						(This is not so much believed now.)
				\end{itemize}
		\end{description}
	\item Latency --- 6 to puberty. Repressed sexual feelings.
	\item Genital --- after puberty. Maturation of sexual interests.
\end{enumerate}

Generally, people do not get through these processes unscathed.

\begin{table}[h]
	\begin{center}
		\begin{tabular}{|p{0.2\textwidth}|p{0.4\textwidth}|p{0.4\textwidth}|}
			\hline
			Stage &Gratification & Fixation \\
			\hline
			Oral & \begin{itemize}
				\item Oral-dependant personality type if we get too much
					gratification during this stage.
				\item Tend to be very dependant and clingy.
			\end{itemize}
			&
			\begin{itemize}
				\item Oral-aggressive
				\item Cynical, cruel, sadistic, verbally agressive.
				\item More likely to over-eat, smoke, kiss a lot.
			\end{itemize}\\
			\hline
			Anal &
			\begin{itemize}
				\item Anal-expulsive
			\end{itemize}
			&
			\begin{itemize}
				\item Anal-retentive, tight-ass.
				\item Compulsively clean
				\item Obsessive, stubborn and rigid.
			\end{itemize}\\
			\hline
			Phallic &
			\multicolumn{2}{p{0.8\textwidth}|}{
			\begin{itemize}
				\item Macho male
				\item Females become very flirtatious.
			\end{itemize}}\\
			\hline
		\end{tabular}
	\end{center}
\end{table}

\subsection{Defense Mechanisms}

\begin{itemize}
	\item Sublimation
	\item Repression
	\item Denial
	\item Displacement
	\item Compensation
\end{itemize}<++>

\end{document}
% vim: tw=80
