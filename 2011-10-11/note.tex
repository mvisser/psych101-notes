\documentclass[12pt]{article}
\usepackage{parskip}
\usepackage{enumerate}
\usepackage{listings}
\usepackage{fullpage}

\begin{document}

\title{PSYCH 101}
\author{Matthew Visser}
\date{Oct 11, 2011}
\maketitle

\textbf{Song of the Day}: \emph{Short Term Memory Blues} by Chris *

\section{Memory}

\begin{itemize}
	\item Memory as Information processing
	\item Effortful memory --- when we are trying to learn and remember.
	\item Memory without awareness --- when we remember but don't realize we
		remember.
	\item Fragility of Memory --- Our memories are often a distorted view of
		reality
\end{itemize}


\subsection{Memory as Information processing}
\subsubsection{Stages of Memory}


See slide on stages of memory.
\begin{itemize}
	\item External events are processed as sensory input.
	\item This goes through sensory memory, then attention to important or novel
		information is encoded as short-term memory.
	\item Short term memories are encoded into long-term memory for storage, and
		decoded back to short-term memory when they are retrieved.
\end{itemize}

To tell a story, you have to retrieve it from long-term memory into short-term
memory, then encode that you told it into short-term then long-term memory.

\subsubsection{Alzheimer's Disease}

Alzheimer's affects short-term memory. The person gets very confused because
they don't remember what they were doing at any point in time. With time, that's
joined by other symptoms such as recognition, speech, remembering how to
manipulate objects.

The disease attacks the hippocampus first, then areas of the brain related to
emotion and reasoning.

What happens in the brain is that tiny molecules are building up in the brain.
These molecules are fine in a single unit, but when it groups together it causes
problem with brain function. This molecule is called Ammeloid-$\beta$.

\subsection{Effortful memory }

\begin{itemize}
	\item How much do we retain?
		\begin{itemize}
			\item we don't retain much in short-term memory.
			\item we forget a lot of what we learn --- we continually forget
				things, and often learn about 20\% in the long-term.
			\item We start to retain things a lot better if we often review
				material.
			\item Repetition is better for remembering things.
			\item We relearn things a lot better than originally learning
				things.
		\end{itemize}
	\item Cues that influence memory are
		\begin{itemize}
			\item Trying to remember things using brute force memorization is
				very ineffective.
			\item The mind works by \emph{associations}
			\item Semantic cues are helpful --- these are  cues that have to do
				with the \emph{meaning} of what you are trying to remember.
			\item Cues related to the self are better than others
			\item Context effects increase memory --- mood and place of memory
				are cues to remember.
		\end{itemize}
	\item Interference in memory --- learning names and learning Spanish after
		learning French.
		\begin{itemize}
			\item Whatever you learn first interferes pro-actively.
			\item What you learn afterward, interferes retro-actively.
		\end{itemize}
	\item Strategies to improve memory
		\begin{itemize}
			\item mnemonics --- creating phrases to help you remember different
				pieces of information.
			\item chunking --- remember just pieces and then remember the
				pieces.
		\end{itemize}
\end{itemize}

\vspace{1cm}
\hrule

\textbf{Second Song of the Day}: \emph{I Am a Glow} by Sarah Harmer

\subsection{Memory without awareness }

\begin{itemize}

	\item How priming can influence memory.
		\begin{itemize}
			\item Priming can bring things out of long-term memory.
		\end{itemize}
	\item Started with amnesiacs
		\begin{itemize}
			\item Some people lose memory through damage to the brain or unknown
				cause.
			\item These people don't remember what happened previously in their
				lives.
			\item These people also have some memory.
			\item Some of the memory is procedural memory
				\begin{itemize}
					\item pro golfer that goes golfing again doesn't remember,
						but when he goes to play golf again, he still has the
						muscle memory.
					\item An amnesiac will not think they have learned how to
						type, but when they try to type, they have learned it
						over time after learning gradually.
					\item Despite having no knowledge that they have learned,
						people can learn.
					\item It doesn't only happen with amnesiacs --- the famous
						names study
					\item sleeper effect --- can't remember the source of a
						memory. Hear a rumour and remember it later but forget
						where you saw it.
				\end{itemize}
		\end{itemize}
\end{itemize}

\subsection{Fragility of Memory }

If memory without awareness can tell us that we know more than we thought,
memory fragility lets us know that much of what we think happened hasn't
actually occurred.

\begin{itemize}
	\item Memory construction --- the construction of automobile destruction.
		\begin{itemize}
			\item started with trying to get eyewitness testimony.
			\item Show people car accidents and ask them slightly different
				questions --- How fast were they going when they bumped into
				each other? Smashed into each other?
		\end{itemize}
	\item We tend to remember what we want to remember.
	\item Can we forget what we want to forget?
		\begin{itemize}
			\item It isn't easy
			\item It is hard to suppress thoughts.
		\end{itemize}
\end{itemize}

\end{document}
% vim: tw=80
