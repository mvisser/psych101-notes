\documentclass[12pt]{article}
\usepackage{parskip}
\usepackage{enumerate}
\usepackage{listings}
\usepackage{fullpage}

\begin{document}

\title{PSYCH 101}
\author{Matthew Visser}
\date{Nov  1, 2011}
\maketitle

\section{Motivation}

Why do we do things?  We have \emph{motivation}. Something causes us to decide
to make a decision todo one thing over another.

\subsection{Mechanistic Approach}

We have a basic model of 

In an environment person $\to$ behaviour $\to$ outcomes.

The person, needs are formed $\to$ behaviour driven.

Freud thought that we have instinctive needs for aggression and sex.

If there is a desireble outcome, it can cause motivation for a behaviour.  An
undesireble outcome recuces the chance of a behaviour.

\subsection{Humanistic Approach}

Maslow formed a theory of a hierarchy of needs. We have needs that are basic to
survival, basic to existence, and then needs that are less necessary such as
psychological needs.

\begin{itemize}
	\item The problems with the hierarchy is that it has been disproven.
	\item Harlow proved that psychological needs are as basic as the need for
		food.
	\item Our needs are not a hierarchy.
\end{itemize}

\subsection{Expectancy}

Murray created the \emph{Expectancy-Value Theory}. The theory identifies some
psychosocial needs.

\hrule

\begin{enumerate}
	\item Have skills
	\item Can execute in Environment
\end{enumerate}

We have four sources of
\begin{enumerate}
	\item performance accomplishments
	\item vicarious experience
	\item social persuasion
	\item emotional arousal
\end{enumerate}

Motivations can be
\begin{itemize}
	\item Additive --- adding motivations just increase the motivations.
	\item undermining --- some motivations undermine the intrinsic value of
		other motivations.
\end{itemize}

\end{document}
% vim: tw=80
